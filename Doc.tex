\documentclass{article}

% NewJeff: I added ErrorBar*.m but there is no demo, no example, and no documentation--
% just a few comments in the ErrorBarManager.

\usepackage[parfill]{parskip}  %  Include a blank line between paragraphs.

\sloppy
\pretolerance=5000
\tolerance=5000
\hyphenpenalty=10000
\exhyphenpenalty=10000
%                                          APA     My max   My min
%                                         default    page     page
\setlength{\footnotesep}    {   5mm}    %  5mm        5mm      5mm
\setlength{\topmargin}      { -15mm}    %  0mm      -25mm    -15mm
\setlength{\oddsidemargin}  {  -6mm}    %  0.25in    -6mm     -6mm
\setlength{\evensidemargin} {  -6mm}    %  0.25in    -6mm     -6mm
\setlength{\textwidth}      { 165mm}    %  6in      165mm    165mm
\setlength{\textheight}     { 240mm}    %  8.5in    240mm    240mm
\setlength{\headheight}     {   3mm}    %  ??         3mm      3mm
\setlength{\headsep}        {  10mm}    %  ??        10mm     10mm
\setlength{\footskip}       {  10mm}    %  ??        10mm     10mm


\title{PlotTbl: Plot Data Stored in a Table}
\author{Jeff Miller (miller at psy.otago.ac.nz)}

\newcommand{\off}[1]{}
\newcommand{\example}[1]{Example: {\it #1}}
\newcommand{\namevalue}[2]{{\it #1 --- #2}}
%\newcommand{\backslash}{\char`\\}
\newcommand{\caret}{\char`\^}

\newcommand{\designelement}[5]{
% LineType
% line type
% cell array of line type specifications
% \{':' '-.'\}   % Example specifications
% the first line will be dotted and the second line will be dash/dot.

\subsection{Name-value pairs for #2 control}

Name-value pair: \namevalue{#1CodeVar}{variable name}  \hspace{6mm} (or equivalently) \\
Name-value pair: \namevalue{#1}{variable name}

This name-value pair indicates that a different #2 should be used
for each different value of the indicated table variable.
The variable name is specified as a character vector, say sVar.

\example{PlotTbl(...,'#1CodeVar','Gender')}

In this example different #2s will be used to plot the data for males versus females (i.e., Gender==1 versus Gender==2).

Name-value pair: \namevalue{#1XVars}{cell array of variable names}  \hspace{6mm} (or equivalently) \\
Name-value pair: \namevalue{#1X}{cell array of variable names}

This name-value pair indicates that different #2s should be used to distinguish
different variables in the table, with those variables being used as X's in the X/Y plots.

\example{PlotTbl(...,'#1XVars',\{'AvgHeight' 'AvgWeight'\})}

In this example different #2s will be used to distinguish lines whose X values
are AvgHeight versus lines whose X values are AvgWeight.

Name-value pair: \namevalue{#1YVars}{cell array of variable names}  \hspace{6mm} (or equivalently) \\
Name-value pair: \namevalue{#1Y}{cell array of variable names}

This name-value pair indicates that different #2s should be used to distinguish
different variables in the table, with those variables being used as Y's in the X/Y plots.

\example{PlotTbl(...,'#1YVars',\{'AvgHeight' 'AvgWeight'\})}

In this example different #2s will be used to distinguish lines whose Y values
are AvgHeight versus lines whose Y values are AvgWeight.

Name-value pair: \namevalue{#1XYVars}{cell array of pairs of variable names}  \hspace{6mm} (or equivalently) \\
Name-value pair: \namevalue{#1XY}{cell array of pairs of variable names}

This name-value pair indicates that different #2s should be used to distinguish
different combinations of variables in the table.
Two variables are listed; the first is used as X's in the X/Y plots, and the second is used as Y's.

\example{PlotTbl(...,'#1XYVars',\{'AvgHeight' 'AvgWeight' 'AvgWeight' 'AvgHeight'\})}

In this example different #2s will be used to distinguish plots of AvgHeight as X and
AvgWeight as Y, as compared with plots of AvgWeight as X and AvgHeight as Y.

Note: You should never specify \emph{more than one} of the four options
#1CodeVar, #1XVars, #1YVars, and #1XYVars.
That is, #2s can only be distinguished in one of these ways---not several.
If you want the plots to be distinguished in more than one way, use other
design elements.


Name-value pair: \namevalue{#1Specs}{#3}

This name-value pair indicates that you want to replace PlotTbl's default set of #2
specifications with a new set.
Use MATLAB's standard #2 specifications in whatever order you prefer.
The #2s will be used in order by values of the CodeVar
or the order of the variables listed for #1XVars or #1YVars.

\example{PlotTbl(...,'#1Specs',#4)}

In this example #5

Name-value pair: \namevalue{#1Order}{'stable' or 'sorted'}

This name-value pair is only applicable when used in conjunction with the '#1CodeVar' pair.
It indicates whether you want PlotTbl to assign the different values of the indicated code variable to the
different #2s in the order in which they appear in the table (i.e., 'stable', which is
the default) or whether you want them assigned in sorted numerical order.

\example{PlotTbl(...,'#1Order','sorted')}

In this example the values of the indicated code variable will be
assigned to #2s in numerical order, regardless of the order in which
they appear in the table.

Name-value pair: \namevalue{#1Legend}{cell array of legend labels}

This name-value pair indicates that you want to replace PlotTbl's default labels of the #2s
with the new labels indicated in your cell array.
The order of the legend labels in the cell array should correspond to the
order in which the different values of the indicated variable are assigned to the different #2s,
as determined by the #1Order name-value pair.

\example{PlotTbl(...,'#1Legend',\{'Male' 'Female'\})}

In this example the legends for the first and second #2s will be ``Male''
and ``Female'', respectively, rather than the defaults of ``Gender=1'' and ``Gender=2''.
Obviously, this option would make sense if used in combination with the name-value pair ``'#1CodeVar','Gender'''.

} % end definition of \designelement

\begin{document}

\maketitle
\tableofcontents

\section{Overview}

PlotTbl is a general-purpose function for creating a figure from data stored in a table.
It was designed to be used with tables in which some variables hold the X/Y values to be plotted and
other variables hold numeric codes that distinguish different conditions.
With such data, PlotTbl makes it easy to plot X/Y relationships for each condition
separately, with different conditions distinguished by
different line types (e.g., solid/dashed), marker types, colors, subplots, etc.
Many options are provided for control over titles, X and Y axis labels, legends, etc.

Screenshots in the README.md file on GitHub show a set of example figures produced with a data table.
All but one of the figures were produced by a single PlotTbl function call.

Users needing even more fine-grained control over individual subplots can call the
function SubplotTbl, which plots only one subplot within a subplot-type figure.

\section{Requirements}

You need a version of MATLAB that supports the ``table'' data type (R2013b or newer).
You also need my MATLAB package ExtractNameVal available at
https://github.com/milleratotago/ExtractNameVal


\section{License}

Copyright (C) 2017--2018, Jeff Miller

This program is free software: you can redistribute it and/or modify
it under the terms of the GNU General Public License as published by
the Free Software Foundation, either version 3 of the License, or
(at your option) any later version.

This program is distributed in the hope that it will be useful,
but WITHOUT ANY WARRANTY; without even the implied warranty of
MERCHANTABILITY or FITNESS FOR A PARTICULAR PURPOSE.  See the
GNU General Public License for more details.

You should have received a copy of the GNU General Public License
along with this program.  If not, see http://www.gnu.org/licenses/ 

\section{Syntax}

PlotTbl is called with one of these four basic parameter configurations:
\begin{enumerate}
\item PlotTbl(Tbl,Name,Value);
\item PlotTbl(Tbl,sX,sY,Name,Value);
\item PlotTbl(Tbl,sX,Name,Value);
\item PlotTbl(Tbl,sY,Name,Value);
\end{enumerate}

Tbl is the data table containing the data to be plotted.

sX and sY are strings indicating the names of the X and Y variables to be plotted.
In some cases the X and/or Y variable names are specified by name-value pairs (see various
option names ending in XVars
and YVars below), and in these cases the sX and/or sY parameters \emph{must} be omitted.

Name,Value represents a series of name-value pair arguments.

\section{Examples}

Suppose that you have a table AHWDat containing the average heights and weights of male
and female children of three different nationalities in six different age groups (see Demo.m).
Here is the format of the table (the heights and weights are not intended to be realistic):

\begin{tabular}{cccccc}
Age & Gender & Nationality & AvgHeight & AvgWeight \\
  2 &    1   &       1     &    114    &   66  \\
  2 &    1   &       2     &    116    &   64  \\
  2 &    1   &       3     &    111    &   68  \\
  2 &    2   &       1     &    125    &   73  \\
  2 &    2   &       2     &    124    &   75  \\
  2 &    2   &       3     &    122    &   75  \\
  4 &    1   &       1     &    117    &   72  \\
  4 &    1   &       2     &    118    &   72  \\
... &    ... &      ...    &    ...    &  ...  \\
\end{tabular}

AHWDat.Gender is 1/2 to code male/female, and AHWDat.Nationality is 1/2/3 to code three nationalities
(e.g., Germany, France, Spain), and AHWDat.Age holds the ages (2, 4, 6, 8, 10, 12).
The variables AHWDat.AvgWeight, and AHWDat.AvgHeight hold the average
weights and heights of the children of each gender, nationality, and age group.
%Age   Gender   Nationality  AvgHeight   AvgWeight

\example{PlotTbl(AHWDat,'Age','AvgWeight','SubplotRowsCodeVar','Gender','SubplotColsCodeVar','Nationality')}

This command produces a figure with six subplots (2 rows x 3 columns).
The two rows of subplots correspond to genders 1 and 2,
and the three columns of subplots correspond to the three nationalities.
Each plot has Age on the X axis and AvgWeight on the Y axis.

\example{PlotTbl(AHWDat,'Age','SubplotRowsYVars',\{'AvgWeight','AvgHeight'\},... \\
'SubplotColsCodeVar','Nationality','LineTypeCodeVar','Gender')}

This command produces a figure with 6 subplots of AvgWeight versus Age in the top row and
AvgHeight versus Age in the bottom row.  The three nationalities are the three columns of subplots.
There are two lines per subplot---one for each Gender---with the lines distinguished by
line type (i.e., solid, dashed).

\example{PlotTbl(AHWDat,'Age','SubplotRowsYVars',\{'AvgWeight','AvgHeight'\},... \\
'SubplotColsCodeVar','Nationality','LineTypeCodeVar','Gender','LineTypeLegend',\{'Male','Female'\})}

This command produces the same plot as the previous one except that the two Gender values are specified
as ``Male'' and ``Female'' in the legend.

For more examples, see Demo.m.

\section{Name-value Pairs for Design Element Control}

It is convenient to consider the name-value pairs as falling into three groups: design element control,
axis control, and legend control.
The largest and most important group provides design element control.

There are seven sets of name-value pair options to control the organization of
the figure in terms of its design elements, namely:
\begin{itemize}
\item line types, such as solid versus dotted (LineType)
\item marker types, such as squares versus circles (MarkerType)
\item color (Color)
\item line width (LineWidth)
\item marker size (MarkerSize)
\item subplot rows (SubplotRows)
\item subplot columns (SubplotCols)
\end{itemize}
All seven are controlled with basically the same set of options.

%These are explained separately, but the most detail is provided for the
%first (i.e., line type control).
%For examples of how the options work together when more than one
%is specified, see Demo.m.

\designelement{LineType}{line type}{cell array of line type specifications}
{\{':' '-.'\}}{the first line will be dotted and the second line will be dash/dot.}

{\bf Note 1:} It is sometimes frustrating that MATLAB only provides four line types.
To work around this limitation, PlotTbl defines some additional line types that
have the same style as the built-in line types (i.e., solid, dashed, etc)
but vary in line thickness.
Specifically, these added line types are specified by strings just like
the additional ones but with the character '2', '3', or '4' added on at the end,
with thickness increasing according to the size of this number.
These added line types are used by default in PlotTbl, so if you need
(say) six line types you will get regular MATLAB ones for the first four
and then you will get thicker versions for the next two.

{\bf Note 2:} Name-value pairs analogous to LineTypeCodeVar, LineTypeXVars, LineTypeYVars,
LineTypeSpecs, and LineTypeLegend, are available for controlling marker types, colors,
line widths, marker sizes, and subplots as described in the following subsections.

\designelement{MarkerType}{marker type}{string list of marker type specifications}
{'so*'}{the first marker will be the square, the second will be the circle, and the third will be the asterisk.}

\designelement{Color}{color}{string list of color specifications}
{'rgbk'}{the first line will be red, the second green, the third blue, and the fourth black.}

{\bf Specifying your own RGB values.} Alternatively, you may specify colors like this:

Name-value pair: \namevalue{ColorSpecs}{cell array of RGB color specifications}

This name-value pair indicates that you want to replace PlotTbl's default set of color
specifications with a new set.
Use MATLAB's standard RGB color specifications in whatever order you prefer.
Your colors will be used in order by values of the CodeVar
or the order of the variables listed for ColorXVars or ColorYVars.

\example{PlotTbl(...,'ColorSpecs',\{[0.4 0.4 0.4], [0.6 0.6 0.6], [0.8 0.8 0.8]\})}

In this example the first line will be the darkest, the second medium, and the third lightest.


\designelement{LineWidth}{line width}{vector of line widths}
{[3 5 8 12]}{the first line will have width 3, the next width 5, etc.}

\designelement{MarkerSize}{marker size}{vector of marker sizes}
{[7 10]}{the first marker will have size 7 and the second will have size 10.}

\subsection{Name-value pairs for subplot row and column control}

Name-value pair: \namevalue{SubplotRowsCodeVar}{variable name} \\
Name-value pair: \namevalue{SubplotColsCodeVar}{variable name}

Each of these two name-value pairs indicates that a different subplot row or subplot
column should be used for each different value of the indicated variable.
The variable name is specified as a character vector, say sVar.

\example{PlotTbl(...,'SubplotRowsCodeVar','Gender')}

This name-value pair indicates that the data for males and females should be plotted in two
different rows of subplots.
The figure might have just one column, or there might be several columns distinguished by the
values of some other variable or by the variable plotted on the X or Y axis.

Name-value pair: \namevalue{SubplotRowsXVars}{cell array of variable names} \\
Name-value pair: \namevalue{SubplotColsXVars}{cell array of variable names}

Each of these two name-value pairs indicates that a different subplot row or column should
be used for X/Y plots with each of the indicated variable names on the X axis.

Name-value pair: \namevalue{SubplotRowsYVars}{cell array of variable names} \\
Name-value pair: \namevalue{SubplotColsYVars}{cell array of variable names}

Each of these two name-value pairs indicates that a different subplot row or column should
be used for X/Y plots with each of the indicated variable names on the Y axis.

Note: You should never specify \emph{more than one} of the four options
SubplotRowsCodeVar, SubplotRowsXVars, SubplotRowsYVars, and SubplotRowsXYVars,
and never more than one of the four corresponding SubplotCols options.
But you can specify both SubplotRows and SubplotCols in any
combination that you want---that is, these ``design elements''
can be controlled independently.

%\namevalue{SubplotRowsSpecs}{unused}
%\namevalue{SubplotColsSpecs}{unused}
%Note: Extrapolating from the previous options, you might expect there to be
%SubplotRowsSpecs and SubplotColsSpecs options, but these do not exist.

Name-value pair: \namevalue{SubplotRowsLegend}{cell array of legend labels} \\
Name-value pair: \namevalue{SubplotColsLegend}{cell array of legend labels}

Each of these two name-value pairs indicates that you want to replace the default labels
of these subplots with the new labels indicated in your cell array.
(Strictly speaking, these are not legends, but rather titles at
the tops of the subplots.)

\example{PlotTbl(...,'SubplotRowsLegend',\{'Male' 'Female'\})}

In this example the first subplot will be labelled ``Male''
and the second will be labelled ``Female'', rather than the defaults of ``Gender=1'' and ``Gender=2''.
This would make sense in combination with the option
\example{PlotTbl(...,'SubplotRowsCodeVar','Gender')}

\section{Name-value Pairs for Control of Axis Labels}

Name-value pair: \namevalue{XLabel}{vector of subplot numbers} \\
Name-value pair: \namevalue{YLabel}{vector of subplot numbers}

Each of these two name-value pairs indicates that you want an X or Y axis label to appear
on the subplot corresponding to one of the numbers in the vector.
By default, all axes are labelled; this option is used when you want to turn off
some of the labels.
The subplots are numbered using the standard numbering scheme for MATLAB's subplot command.

\example{PlotTbl(...,'XLabel',[7 8 9],'YLabel',[1 4 7])}

In this example, with a 3x3 arrangement of subplots, the X axes will
be labelled for the plots in the bottom row (i.e., subplot numbers 7-9) and the Y axes
will be labelled for the subplots in the left-most column (i.e., subplot numbers 1, 4, and 7).

Name-value pair: \namevalue{XLabelStr}{cell array of X labels for the subplots} \\
Name-value pair: \namevalue{YLabelStr}{cell array of Y labels for the subplots}

Each of these two name-value pairs allows you to specify the text of the X and Y axis labels for the
subplots, which can be used to override the defaults (i.e., the X and Y variable names).
This capability is needed when the desired axis label is not a legal MATLAB variable name.
The length of the cell array is the number of subplots (i.e., you must specify
one X or Y axis label for each subplot).

\example{PlotTbl(...,'XLabelStr',\{'\textbackslash alpha' '\textbackslash alpha' '\textbackslash alpha' '\textbackslash alpha'\})}

In this example the X axes of all four subplots will be labelled $\alpha$.

\section{Name-value Pairs for Control of Legends}

Name-value pair: \namevalue{Legend}{vector of subplot numbers}

This name-value pair provides a list of numbers of the subplots for which you want the legends to be displayed.
By default, a legend is displayed only on the first subplot.

\example{PlotTbl(...,'Legend',[1 3])}

In this example legends will be displayed on the first and third subplots.

Name-value pair: \namevalue{LegendLoc}{string}

This name-value pair can be used to override MATLAB's default choice of where to
put the legend(s) on the plot(s).
Use MATLAB's standard location specifications (e.g., 'Northeast', 'Best')
to indicate where you want the legend(s).

Name-value pair: \namevalue{LegendPos}{vector of four position numbers}

Alternatively, this name-value pair can be used to override MATLAB's default choice of where to
put the legend(s) on the plot(s).
The option indicates that the legend(s) should be placed in the custom position
indicated by the four numerical values in the vector (i.e., left bottom width height).
Use MATLAB's standard position specification, determined by its units
(e.g., see https://www.mathworks.com/matlabcentral/answers/ 80980-what-does-the-four-vector-mean-in-matlab-legend-position ).

Note: To override the default legend positioning, you can specify LegendLoc or LegendPos, but not both.

Name-value pair: \namevalue{LegendBox}{'boxon' or 'boxoff'}

This name-value pair indicates whether you want the box on the legend(s).
I think boxes just add clutter, so the default is 'boxoff'.

\section{Name-value Pairs for Adding Reference Lines}  % NEWJEFF: This section is new.

Name-value pair: \namevalue{AddXRef}{real number}

This name-value pair adds a vertical reference line at the indicated X value;
for example, you might want to add a reference line at X=0 or at X=the average X value.

By default this reference line is black and dotted, but you can change these characteristics
with the following:

Name-value pair: \namevalue{XRefStyle}{character string}

The character string of this pair is used as a MATLAB line descriptor such as 'k:' for a
black dotted line (see appendices for lists of MATLAB's line types and colors).

Name-value pair: \namevalue{AddYRef}{real number}

Name-value pair: \namevalue{YRefStyle}{character string}

These are analogous to AddXRef and XRefStyle to add a horizontal line at the indicated Y value.

Name-only option: \namevalue{AddDiagonal}{ }

This option has no argument, but merely indicates that a diagonal line should be drawn to mark X=Y.
The following pair controls the style of this line, analogous to XRefStyle and YRefStyle.

Name-value pair: \namevalue{DiagonalStyle}{character string}

\section{SubplotReshape}

It is sometimes desirable to override PlotTbl's default configuration of
the subplots within a figure.
For example, suppose you want to display subplots showing 9 different Y variables.
If you simply use SubplotRowsYVars or SubplotColsYVars, you will get a 9x1
or 1x9 configuration of plots, which does not look good.
A 3x3 layout would look much better, and you can request that with
the SubplotReshape name-value pair.

Name-value pair: \namevalue{SubplotReshape}{[nrows ncols]}

\example{PlotTbl(...,'SubplotReshape',[3 3])}

In this example, PlotTbl will arrange the subplots in a configuration with
3 rows and 3 cols.
(Note that a 3x3 layout could also be used even if there were only 8 different
Y variables; in that case, there would just be an empty cell in the matrix of
subplots.)

\section{Customizing Further}  % NEWJEFF: New section

MATLAB's plot command recognizes many ``extra'' options that are not explicitly handled by PlotTbl.
You can use the PassThru command to set these extra options:

Name-value pair: \namevalue{PassThru}{cell array of parameters to pass to plot}

\example{PlotTbl(...,'PassThru',{'MarkerIndices',[1 5 10]})}

In this example, PlotTbl will display markers for the 1st, 5th, and 10th data points.

For further control over the plots, you can pass a function for PlotTbl to
call after each subplot is plotted.

This function is called with the subplot row and column numbers as arguments,
so you can do different things for different subplots if you want.

Name-value pair: \namevalue{Customize}{YourFunctionName}

For example, you could define a function to set the horizontal and vertical
axes to the ranges of (0,1):

\example{MyCustomFn = @(x,y)(axis([0 1 0 1])); \% Define this custom function before calling PlotTbl}

Then you would tell PlotTbl to call it after each subplot like this:

\example{PlotTbl(...,'Customize',MyCustomFn)}

Alternatively, the parameter following 'Customize' can be a cell array of function handles
if you want to call several functions to customize each graph.
For example, here are three functions to set the axis ranges and to specify the tic values and for the X axis:

% NEWJEFF: Maybe I need to add specific commands to control xtics and xticklabels?
\example{MyCustomAxFn = @(x,y)(axis([0.5 2.5 60 100]));}
\example{MyCustomXTicFn = @(x,y)(xticks([1 2]));}
\example{MyCustomXTicLblFn = @(x,y)(xticklabels({'far', 'near'}));}

Then you would tell PlotTbl to call these three functions after each subplot by including them
in a cell array, like this:

\example{PlotTbl(...,'Customize',{MyCustomAxFn,MyCustomXTicFn,MyCustomXTicLblFn,})}

Finally, even more fine-grained control can be achieved by calling the function SubplotTbl directly.
SubplotTbl recognizes nearly all of the same formatting commands as PlotTbl
except for SubplotRows and SubplotCols, which you have to manage yourself.
See Demo.m for an example.

\section{Tips}

\begin{itemize}

\item PlotTbl plots the data from all of the rows in whatever table is passed to it.
If you just want to plot some of the rows of a given table, just pass
the desired row subset using MATLAB's standard row-selection methods.

\example{PlotTbl(AHWDat(AHWDat.Gender==2,:),'Age','AvgWeight','LineTypeCodeVar','Nationality');} 

In this example, data are just plotted for Gender 2.

\item When you select out certain rows as suggested in the preceding tip, remember that
this can change the order in which various CodeVar values appear in the table.
For example, if you created a new table by eliminating rows with small values of height, the first gender
to appear in the new table might be female, even though the males appeared first in the original table.
This can create problems when you use the 'stable' Order and specify your own legends
with a command like 'LineTypeLegend',\{'Male' 'Female'\}.
If the females appear first in the new table (after selection), then the male and
female legend labels will be reversed---not good!

A useful trick to avoid this problem is to reset the values that you do not want plotted to NaN.
All rows then remain in the same so the orders of CodeVars are not perturbed,
but MATLAB does not plot the NaN values so they are omitted, as you want.



\item For any design element that does \emph{not} vary across your plots,
PlotTbl uses the first of the default ``Specs'' options for that design element.
If you would prefer something else, you can change this just by setting the Specs
for that design element, even though you are not using it to distinguish between lines.
You can include the Specs option for any design element, whether that element appears
anywhere else in the PlotTbl command or not.

\example{PlotTbl(...,'MarkerTypeSpecs',' ',...);} 

In this example, all lines will be drawn without markers (as indicated by the blank marker type specification).
By default, PlotTbl's first MarkerTypeSpec is the square, so the lines would be drawn with square
markers if this option were not specified.

\item After plotting, it is possible to change the titles shown on the subplots to any titles that
you want.  See Demo.m for an example.

\item PlotTbl repeats specifications if necessary.  For example, MATLAB defines 4 line types (see Appendix).
If you want to distinguish more than 4 categories via line types, PlotTbl will recycle through the same ones.
This is not ideal, but perhaps better than bombing.

\item MATLAB leaves a lot of space between subplots---often, too much for my taste.
The function 'subplotResize' can be used to stretch the subplots, horizontally and vertically,
into the space that MATLAB would normally leave between them.

\item The function CellArModify often provides a convenient way to modify the properties of
all subplots.  See Demo.m for an example.

\end{itemize}

\appendix

% \clearpage

\section{Appendix: Complete List of Name-value Pairs}

% Don't sort this list: Legend, XLabel, and YLabel go to the ends.
\namevalue{AddDiagonal}{(takes no parameter)} \\  % NEWJEFF: Next 6 lines are new
\hspace*{1cm} \namevalue{DiagonalStyle}{string line descriptor; default is 'k:'} \\
\namevalue{AddXRef}{real number at which to add a vertical reference line} \\
\hspace*{1cm} \namevalue{XRefStyle}{string line descriptor; default is 'k:'} \\
\namevalue{AddYRef}{real number at which to add a horizontal reference line} \\
\hspace*{1cm} \namevalue{YRefStyle}{string line descriptor; default is 'k:'} \\
\namevalue{ColorCodeVar}{variable name} \\
\namevalue{ColorLegend}{cell array of legend labels} \\
\namevalue{ColorOrder}{'stable' or 'sorted'} \\
\namevalue{ColorSpecs}{string list of colors, or cell array of RGB triples} \\
\namevalue{ColorXVars}{cell array of variable names} \\
\namevalue{ColorYVars}{cell array of variable names} \\
\namevalue{Legend}{vector of subplot numbers where you want the legend} \\
\namevalue{LegendBox}{'boxon' or 'boxoff'} \\
\namevalue{LegendLoc}{string such as 'Northeast'} \\
\namevalue{LegendPos}{[left bottom width height]} \\
\namevalue{LineTypeCodeVar}{variable name} \\
\namevalue{LineTypeLegend}{cell array of legend labels} \\
\namevalue{LineTypeOrder}{'stable' or 'sorted'} \\
\namevalue{LineTypeSpecs}{cell array of line type specifications} \\
\namevalue{LineTypeXVars}{cell array of variable names} \\
\namevalue{LineTypeYVars}{cell array of variable names} \\
\namevalue{LineWidthCodeVar}{variable name} \\
\namevalue{LineWidthLegend}{cell array of legend labels} \\
\namevalue{LineWidthOrder}{'stable' or 'sorted'} \\
\namevalue{LineWidthSpecs}{vector of line widths} \\
\namevalue{LineWidthXVars}{cell array of variable names} \\
\namevalue{LineWidthYVars}{cell array of variable names} \\
\namevalue{MarkerSizeCodeVar}{variable name} \\
\namevalue{MarkerSizeLegend}{cell array of legend labels} \\
\namevalue{MarkerSizeOrder}{'stable' or 'sorted'} \\
\namevalue{MarkerSizeSpecs}{vector of marker sizes} \\
\namevalue{MarkerSizeXVars}{cell array of variable names} \\
\namevalue{MarkerSizeYVars}{cell array of variable names} \\
\namevalue{MarkerTypeCodeVar}{variable name} \\
\namevalue{MarkerTypeLegend}{cell array of legend labels} \\
\namevalue{MarkerTypeOrder}{'stable' or 'sorted'} \\
\namevalue{MarkerTypeSpecs}{string list of marker type specifications} \\
\namevalue{MarkerTypeXVars}{cell array of variable names} \\
\namevalue{MarkerTypeYVars}{cell array of variable names} \\
\namevalue{SubplotColsCodeVar}{variable name} \\
\namevalue{SubplotColsLegend}{cell array of legend labels} \\
\namevalue{SubplotColsOrder}{'stable' or 'sorted'} \\
\namevalue{SubplotColsXVars}{cell array of variable names} \\
\namevalue{SubplotColsYVars}{cell array of variable names} \\
\namevalue{SubplotReshape}{[nrows ncols]} \\
\namevalue{SubplotRowsCodeVar}{variable name} \\
\namevalue{SubplotRowsLegend}{cell array of legend labels} \\
\namevalue{SubplotRowsOrder}{'stable' or 'sorted'} \\
\namevalue{SubplotRowsXVars}{cell array of variable names} \\
\namevalue{SubplotRowsYVars}{cell array of variable names} \\
\namevalue{XLabel}{vector of subplot numbers} \\
\namevalue{XLabelStr}{cell array of X labels for the subplots} \\
\namevalue{YLabel}{vector of subplot numbers} \\
\namevalue{YLabelStr}{cell array of Y labels for the subplots} \\

Note: Name-value pairs of the form ``???CodeVar'' can be abbreviated as just ``???'',
and those of the form ``???XVars'' or ``???YVars'' can be abbreviated as ``???X'' or ``???Y''.

\section{Appendix: Line Types in MATLAB}

In PlotTbl's default order:

\begin{itemize}
\item '-': Solid line
\item '-{}-': Dashed line  % Note the curly brackets to prevent conversion to a single long dash.
\item ':': Dotted line
\item '-.': Dash-dot line
\end{itemize}

Use 'none' or ' ' to omit the line.
% NEWJEFF: For additional options, see the FileExchange contribution dashLine.
% But I am not sure how these could be used in PlotTbl.

\section{Appendix: Marker Types in MATLAB}

In PlotTbl's default order:

\begin{itemize}
\item 'square' or 's': Square
\item 'o': Circle
\item 'diamond' or 'd': Diamond
\item '\caret': Upward-pointing triangle
\item 'v': Downward-pointing triangle
\item '+': Plus sign
\item 'x': Cross
\item '*': Asterisk
\item '.': Point
\item '$>$': Right-pointing triangle
\item '$<$': Left-pointing triangle
\item 'pentagram' or 'p': Five-pointed star (pentagram)
\item 'hexagram' or 'h': Six-pointed star (hexagram)
\end{itemize}

Use 'none' or ' ' to omit the marker.

\section{Appendix: Colors in MATLAB}

In PlotTbl's default order:

\begin{itemize}
\item k: Black
\item r: Red
\item g: Green
\item b: Blue
\item c: Cyan
\item m: Magenta
\item y: Yellow
\item w: White
\end{itemize}

\end{document}
