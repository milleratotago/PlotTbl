\documentclass{article}

\title{PlotTbl: Plot the data stored in a table}
\author{Jeff Miller}

\newcommand{\off}[1]{}
\newcommand{\example}[1]{\mbox{ } \\ Example: {\it #1} \\ }
\newcommand{\namevalue}[2]{\mbox{ } \\ Name-value pair: {\it #1 --- #2}}
%\newcommand{\backslash}{\char`\\}
\newcommand{\caret}{\char`\^}

\newcommand{\designelement}[5]{
% LineType
% line type
% cell array of line type specifications
% \{':' '-.'\}   % Example specifications
% the first line will be dotted and the second line will be dash/dot.

\subsection{Name-value pairs for #2 control}

\namevalue{#1CodeVar}{variable name}

This name-value pair indicates that a different #2 should be used
for each different value of the indicated variable.
The variable name is specified as a character vector, say sVar.

\example{PlotTbl(...,'#1CodeVar','Gender')}

In this example different #2s will be used to plot the data for males versus females.

\namevalue{#1XVars}{cell array of variable names}

This name-value pair indicates that different #2s should be used to distinguish
different variables in the table, with those variables being used as X's in the X/Y plots.

\example{PlotTbl(...,'#1XVars',\{'AvgHeight' 'AvgWeight'\})}

In this example different #2s will be used depending on whether the X values
are AvgHeight or AvgWeight.

\namevalue{#1YVars}{cell array of variable names}

This name-value pair indicates that different #2s should be used to distinguish
different variables in the table, with those variables being used as Y's in the X/Y plots.

\example{PlotTbl(...,'#1YVars',\{'AvgHeight' 'AvgWeight'\})}

In this example different #2s will be used depending on whether the Y values
are AvgHeight or AvgWeight. \\

Note: You should never specify \emph{more than one} of the three options
#1CodeVar, #1XVars, and #1YVars.
That is, #2s can only be distinguished in one way or the other---not several.
If you want the #2s to be distinguished in more than one way, use other
design elements.

\namevalue{#1Specs}{#3}

This name-value pair indicates that you want to replace the default set of #2
specifications with a new set.
Use MATLAB's standard #2 specifications in whatever order you prefer.
The #2s will be used in order by values of the CodeVar
or the order of the variables listed for #1XVars or #1YVars.

\example{PlotTbl(...,'#1Specs',#4)}

In this example #5

\namevalue{#1Legend}{cell array of legend labels}

This name-value pair indicates that you want to replace the default labels of the #2s
with the new labels indicated in your cell array.

\example{PlotTbl(...,'#1Legend',\{'Male' 'Female'\})}

In this example the legends for the first and second #2s will be ``Male''
and ``Female'', respectively.
This would make sense if used in combination with the name-value pair '#1CodeVar','Gender'.

} % end definition of \designelement

\begin{document}

\maketitle
\tableofcontents

\section{Overview}

PlotTbl is a general-purpose function for creating a figure from data stored in a table.
It was designed to be used with tables in which some variables hold the X/Y values to be plotted and
other variables hold numeric codes that distinguish different conditions.
With such data, PlotTbl makes it easy to make plots showing X/Y relationships
for each condition separately, with different conditions distinguished by being in different subplots,
different line types (e.g., solid/dashed), different marker types, different colors, etc.
Some options are provided for control over titles, X and Y axis labels, legends, etc.

The screenshots on GitHub show an example data table and a set of figures produced
from it by PlotTbl.
All but one of the figures---the last one---were produced by a single PlotTbl function call.

Users needing even more fine-grained control over individual subplots can call the
function SubplotTbl, which plots only one subplot within a subplot-type figure.

\section{Requirements}

You need a version of MATLAB that supports the ``table'' data type.
You also need my MATLAB package ExtractNameVal, which is available here: \\
https://github.com/milleratotago/ExtractNameVal


\section{License}

Copyright (C) 2017, Jeff Miller

This program is free software: you can redistribute it and/or modify
it under the terms of the GNU General Public License as published by
the Free Software Foundation, either version 3 of the License, or
(at your option) any later version.

This program is distributed in the hope that it will be useful,
but WITHOUT ANY WARRANTY; without even the implied warranty of
MERCHANTABILITY or FITNESS FOR A PARTICULAR PURPOSE.  See the
GNU General Public License for more details.

You should have received a copy of the GNU General Public License
along with this program.  If not, see <http://www.gnu.org/licenses/>.

\section{Syntax}

PlotTbl is called with one of these four basic parameter configurations:
\begin{enumerate}
\item PlotTbl(Tbl,Name,Value);
\item PlotTbl(Tbl,sX,sY,Name,Value);
\item PlotTbl(Tbl,sX,Name,Value);
\item PlotTbl(Tbl,sY,Name,Value);
\end{enumerate}

Tbl is the data table containing the data to be plotted.

sX and sY are strings indicating the names of the X and Y variables to be plotted.
These parameters \emph{must} be omitted when the variable names are specified by name-value pairs
(see XVars and YVars below).

Name,Value represents a series of name-value pair arguments.

\section{Examples}

Suppose that you have a table AHWDat containing the average heights and weights of male
and female children of three different nationalities in six different age groups (see PlotTblDemo.m).
AHWDat.Gender is 1/2 to code male/female, and AHWDat.Nationality is 1/2/3 to code three nationalities
(e.g., Germany, France, Spain), and AHWDat.Age holds the ages (2, 4, 6, 8, 10, 12).
The variables AHWDat.AvgWeight, and AHWDat.AvgHeight hold the average
weights and heights of the children of each gender, nationality, and age group.
%Age   Gender   Nationality  AvgHeight   AvgWeight

\example{PlotTbl(AHWDat,'Age','AvgWeight','SubplotRowsCodeVar',... \\
'Gender','SubplotColsCodeVar','Nationality')}

This command produces a plot with six subplots (2 rows x 3 columns).
The two rows of subplots correspond to genders 1 and 2,
and the three columns correspond to the three nationalities.
Each plot has Age on the X axis and AvgWeight on the Y axis.

\example{PlotTbl(AHWDat,'Age','SubplotRowsYVars',\{'AvgWeight','AvgHeight'\},... \\
'SubplotColsCodeVar','Nationality','LineTypeCodeVar','Gender')}

This command produces a 6-subplot plot with plots of AvgWeight versus Age in the top row and
AvgHeight versus Age in the bottom row.  The three nationalities are the three columns.
There are two lines per subplot---one for each Gender---with the lines distinguished by
line type (i.e., solid, dashed).

\example{PlotTbl(AHWDat,'Age','SubplotRowsYVars',\{'AvgWeight','AvgHeight'\},... \\
'SubplotColsCodeVar','Nationality','LineTypeCodeVar','Gender',... \\
'LineTypeLegend',\{'Male','Female'\})}

This command produces the same plot as the previous one except that the two Gender values are called Male and Female in the legend.

For more examples, see Demo.m.

\section{Name-value pairs for design-element control}

There are seven similar sets of name-value pair options to control the organization of
the figure in terms of its design elements, namely:
\begin{itemize}
\item line types, such as solid versus dotted (LineType)
\item marker types, such as squares versus circles (MarkerType)
\item color (Color)
\item line width (LineWidth)
\item marker size (MarkerSize)
\item subplot rows (SubplotRows)
\item subplot columns (SubplotCols)
\end{itemize}

%These are explained separately, but the most detail is provided for the
%first (i.e., line type control).
%For examples of how the options work together when more than one
%is specified, see Demo.m.

\designelement{LineType}{line type}{cell array of line type specifications}
{\{':' '-.'\}}{the first line will be dotted and the second line will be dash/dot.}

\mbox{ } \\
Note: Name-value pairs analogous to LineTypeCodeVar, LineTypeXVars, LineTypeYVars,
LineTypeSpecs, and LineTypeLegend, are available for controlling marker types, colors,
line widths, and marker sizes, as described next.
There are also very similar options for controlling subplots.

\designelement{MarkerType}{marker type}{string list of marker type specifications}
{'so*'}{the first marker will be the square, the second will be the circle, and the third will be the asterisk.}

\designelement{Color}{color}{string list of color specifications}
{'rgbk'}{the first line will be red, the second green, the third blue, and the fourth black.}

\designelement{LineWidth}{line width}{vector of line widths}
{[3 5 8 12]}{the first line will have width 3, the next width 5, etc.}

\designelement{MarkerSize}{marker sizes}{vector of marker sizes}
{[7 10]}{the first marker will have size 7 and the second will have size 10.}

\subsection{Name-value pairs for subplot row and column control}

\namevalue{SubplotRowsCodeVar}{variable name}
\namevalue{SubplotColsCodeVar}{variable name}

Each of these two name-value pairs indicates that a different subplot row or column
should be used for each different value of the indicated variable.
The variable name is specified as a character vector, say sVar.

\example{PlotTbl(...,'SubplotRowsCodeVar','Gender')}

This name-value pair indicates that the data for males and females should be plotted in two
different rows of subplots.

\namevalue{SubplotRowsXVars}{cell array of variable names}
\namevalue{SubplotColsXVars}{cell array of variable names}

Each of these two name-value pairs indicates that a different subplot row or column should
be used for X/Y plots with each of the indicated variable names on the X axis.

\namevalue{SubplotRowsYVars}{cell array of variable names}
\namevalue{SubplotColsYVars}{cell array of variable names}

Each of these two name-value pairs indicates that a different subplot row or column should
be used for X/Y plots with each of the indicated variable names on the Y axis.

Note: You should never specify \emph{more than one} of the three options
SubplotRowsCodeVar, SubplotRowsXVars, and SubplotRowsYVars,
and never more than one of the three corresponding SubplotCols options.
But you can specify both SubplotRows and SubplotCols in any
combination that you want, since these are independent ``design elements''
of the display (in the same sense that the line type and color are independent).

%\namevalue{SubplotRowsSpecs}{unused}
%\namevalue{SubplotColsSpecs}{unused}
Note: Extrapolating from the previous options, you might expect there to be
SubplotRowsSpecs and SubplotColsSpecs options, but these do not exist.

\namevalue{SubplotRowsLegend}{cell array of legend labels}
\namevalue{SubplotColsLegend}{cell array of legend labels}

Each of these two name-value pairs indicates that you want to replace the default labels
of these subplots with the new labels indicated in your cell array.
(Strictly speaking, these are not legends, but rather titles at
the tops of the subplots.)

\example{PlotTbl(...,'SubplotRowsLegend',\{'Male' 'Female'\})}
This name-value pair indicates that you want the first subplot to be labelled Male
and the second to be labelled Female.
This would make sense in combination with the option
\example{PlotTbl(...,'SubplotRowsCodeVar','Gender')}

\section{Name-value pairs for control of axis labels}

\namevalue{XLabel}{vector of subplot numbers}
\namevalue{YLabel}{vector of subplot numbers}
The length of the cell array is the number of subplots.

\example{PlotTbl(...,'XLabel',[7 8 9],'YLabel',[1 4 7)}
With a 3x3 arrangement of subplots, this would label the X axes
of the plots in the bottom row and it would label the Y axes of
the subplots in the left-most column.

\namevalue{XLabelStr}{cell array of X labels for the subplots}
\namevalue{YLabelStr}{cell array of Y labels for the subplots}
These options specify the X and Y axis labels for the subplots, which can be used to
override the defaults (i.e., the X and Y variable names).
These are essential when the desired axis label is not a legal MATLAB variable name.
The length of the cell array is the number of subplots.

\example{PlotTbl(...,'XLabelStr',\{'\textbackslash alpha' '\textbackslash alpha' '\textbackslash alpha' '\textbackslash alpha'\}}
This indicates that the X axes of all four subplots should be labelled $\alpha$.

\section{Name-value pairs for control of legends}

\namevalue{Legend}{vector of subplot numbers}
A list of the numbers of the subplots on which legends should be displayed.

\example{PlotTbl(...,'Legend',[1:3]}

This indicates that legends should be displayed in the first three subplots.

\namevalue{LegendLoc}{string}

\namevalue{LegendPos}{[left bottom width height]}

\namevalue{LegendBox}{true or false}

This name-value pair indicates whether you want the box on the legend(s).
I think they just add clutter, so the default is false.

\section{Tips}

\begin{itemize}

\item PlotTbl plots the data from all of the rows in your table.
If you just want to plot some of the rows, just pass it a subset
of the table using MATLAB's standard row-selection methods.
In the following example, data are just plotted for the females.

\example{PlotTbl(AHWDat(AHWDat.Gender==2,:),'Age','AvgWeight','LineTypeCodeVar','Nationality');} 


\item For any design-element that does \emph{not} vary across your plots,
PlotTbl uses the first of the default ``Specs'' options for that design element.
If you would prefer something else, you can change this just by setting the Specs
for that design element, even though you are not using it to distinguish between lines.

For example, the square is the first of the default MarkerTypeSpecs, so all of the lines
will be drawn with square markers if you do not specify anything related to MarkerType
(i.e., MarkerTypeCodeVar, MarkerTypeXVars, or MarkerTypeYVars).
Perhaps you would prefer lines with no markers at all (e.g., if the points are closely
spaced along the X axis so that the markers smash together)?
In that case, include the option {\it PlotTbl(...,'MarkerTypeSpecs',' ',...) }
so that the first (and only) element of MarkerTypeSpecs is the space,
which asks for no marker at all.
You can always include the Specs option for any design element,
whether that element appears anywhere else in the PlotTbl command or not.

\end{itemize}

\appendix

\section{Appendix: A complete list of name-value pairs}

\namevalue{ColorCodeVar}{variable name}
\namevalue{ColorLegend}{cell array of legend labels}
\namevalue{ColorSpecs}{string list of colors}
\namevalue{ColorXVars}{cell array of variable names}
\namevalue{ColorYVars}{cell array of variable names}
\namevalue{Legend}{vector of subplot numbers}
\namevalue{LegendBox}{true or false}
\namevalue{LegendLoc}{}
\namevalue{LegendPos}{[left bottom width height]}
\namevalue{LineTypeCodeVar}{variable name}
\namevalue{LineTypeLegend}{cell array of legend labels}
\namevalue{LineTypeSpecs}{cell array of line type specifications}
\namevalue{LineTypeXVars}{cell array of variable names}
\namevalue{LineTypeYVars}{cell array of variable names}
\namevalue{LineWidthCodeVar}{variable name}
\namevalue{LineWidthLegend}{cell array of legend labels}
\namevalue{LineWidthSpecs}{vector of line widths}
\namevalue{LineWidthXVars}{cell array of variable names}
\namevalue{LineWidthYVars}{cell array of variable names}
\namevalue{MarkerSizeCodeVar}{variable name}
\namevalue{MarkerSizeLegend}{cell array of legend labels}
\namevalue{MarkerSizeSpecs}{vector of marker sizes}
\namevalue{MarkerSizeXVars}{cell array of variable names}
\namevalue{MarkerSizeYVars}{cell array of variable names}
\namevalue{MarkerTypeCodeVar}{variable name}
\namevalue{MarkerTypeLegend}{cell array of legend labels}
\namevalue{MarkerTypeSpecs}{string list of marker type specifications}
\namevalue{MarkerTypeXVars}{cell array of variable names}
\namevalue{MarkerTypeYVars}{cell array of variable names}
\namevalue{SubplotColsCodeVar}{variable name}
\namevalue{SubplotColsLegend}{cell array of legend labels}
\namevalue{SubplotColsXVars}{cell array of variable names}
\namevalue{SubplotColsYVars}{cell array of variable names}
\namevalue{SubplotRowsCodeVar}{variable name}
\namevalue{SubplotRowsLegend}{cell array of legend labels}
\namevalue{SubplotRowsXVars}{cell array of variable names}
\namevalue{SubplotRowsYVars}{cell array of variable names}
\namevalue{XLabel}{vector of subplot numbers}
\namevalue{XLabelStr}{vector of subplot numbers}
\namevalue{YLabel}{vector of subplot numbers}
\namevalue{YLabelStr}{vector of subplot numbers}

\section{Appendix: Line types in MATLAB}

In PlotTbl's default order:

\begin{itemize}
\item '-': Solid line
\item '--': Dashed line
\item ':': Dotted line
\item '-.': Dash-dot line
\end{itemize}

\section{Appendix: Marker types in MATLAB}

In PlotTbl's default order:

\begin{itemize}
\item 'square' or 's': Square
\item 'o': Circle
\item 'diamond' or 'd': Diamond
\item '\caret': Upward-pointing triangle
\item 'v': Downward-pointing triangle
\item '+': Plus sign
\item 'x': Cross
\item '*': Asterisk
\item '.': Point
\item '$>$': Right-pointing triangle
\item '$<$': Left-pointing triangle
\item 'pentagram' or 'p': Five-pointed star (pentagram)
\item 'hexagram' or 'h': Six-pointed star (hexagram)
\end{itemize}

\section{Appendix: Colors in MATLAB}

In PlotTbl's default order:

\begin{itemize}
\item k: Black
\item r: Red
\item g: Green
\item b: Blue
\item c: Cyan
\item m: Magenta
\item y: Yellow
\item w: White
\end{itemize}

\end{document}
